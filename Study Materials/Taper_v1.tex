% Options for packages loaded elsewhere
\PassOptionsToPackage{unicode}{hyperref}
\PassOptionsToPackage{hyphens}{url}
%
\documentclass[
]{article}
\usepackage{amsmath,amssymb}
\usepackage{lmodern}
\usepackage{iftex}
\ifPDFTeX
  \usepackage[T1]{fontenc}
  \usepackage[utf8]{inputenc}
  \usepackage{textcomp} % provide euro and other symbols
\else % if luatex or xetex
  \usepackage{unicode-math}
  \defaultfontfeatures{Scale=MatchLowercase}
  \defaultfontfeatures[\rmfamily]{Ligatures=TeX,Scale=1}
\fi
% Use upquote if available, for straight quotes in verbatim environments
\IfFileExists{upquote.sty}{\usepackage{upquote}}{}
\IfFileExists{microtype.sty}{% use microtype if available
  \usepackage[]{microtype}
  \UseMicrotypeSet[protrusion]{basicmath} % disable protrusion for tt fonts
}{}
\makeatletter
\@ifundefined{KOMAClassName}{% if non-KOMA class
  \IfFileExists{parskip.sty}{%
    \usepackage{parskip}
  }{% else
    \setlength{\parindent}{0pt}
    \setlength{\parskip}{6pt plus 2pt minus 1pt}}
}{% if KOMA class
  \KOMAoptions{parskip=half}}
\makeatother
\usepackage{xcolor}
\usepackage[margin=1in]{geometry}
\usepackage{color}
\usepackage{fancyvrb}
\newcommand{\VerbBar}{|}
\newcommand{\VERB}{\Verb[commandchars=\\\{\}]}
\DefineVerbatimEnvironment{Highlighting}{Verbatim}{commandchars=\\\{\}}
% Add ',fontsize=\small' for more characters per line
\usepackage{framed}
\definecolor{shadecolor}{RGB}{248,248,248}
\newenvironment{Shaded}{\begin{snugshade}}{\end{snugshade}}
\newcommand{\AlertTok}[1]{\textcolor[rgb]{0.94,0.16,0.16}{#1}}
\newcommand{\AnnotationTok}[1]{\textcolor[rgb]{0.56,0.35,0.01}{\textbf{\textit{#1}}}}
\newcommand{\AttributeTok}[1]{\textcolor[rgb]{0.77,0.63,0.00}{#1}}
\newcommand{\BaseNTok}[1]{\textcolor[rgb]{0.00,0.00,0.81}{#1}}
\newcommand{\BuiltInTok}[1]{#1}
\newcommand{\CharTok}[1]{\textcolor[rgb]{0.31,0.60,0.02}{#1}}
\newcommand{\CommentTok}[1]{\textcolor[rgb]{0.56,0.35,0.01}{\textit{#1}}}
\newcommand{\CommentVarTok}[1]{\textcolor[rgb]{0.56,0.35,0.01}{\textbf{\textit{#1}}}}
\newcommand{\ConstantTok}[1]{\textcolor[rgb]{0.00,0.00,0.00}{#1}}
\newcommand{\ControlFlowTok}[1]{\textcolor[rgb]{0.13,0.29,0.53}{\textbf{#1}}}
\newcommand{\DataTypeTok}[1]{\textcolor[rgb]{0.13,0.29,0.53}{#1}}
\newcommand{\DecValTok}[1]{\textcolor[rgb]{0.00,0.00,0.81}{#1}}
\newcommand{\DocumentationTok}[1]{\textcolor[rgb]{0.56,0.35,0.01}{\textbf{\textit{#1}}}}
\newcommand{\ErrorTok}[1]{\textcolor[rgb]{0.64,0.00,0.00}{\textbf{#1}}}
\newcommand{\ExtensionTok}[1]{#1}
\newcommand{\FloatTok}[1]{\textcolor[rgb]{0.00,0.00,0.81}{#1}}
\newcommand{\FunctionTok}[1]{\textcolor[rgb]{0.00,0.00,0.00}{#1}}
\newcommand{\ImportTok}[1]{#1}
\newcommand{\InformationTok}[1]{\textcolor[rgb]{0.56,0.35,0.01}{\textbf{\textit{#1}}}}
\newcommand{\KeywordTok}[1]{\textcolor[rgb]{0.13,0.29,0.53}{\textbf{#1}}}
\newcommand{\NormalTok}[1]{#1}
\newcommand{\OperatorTok}[1]{\textcolor[rgb]{0.81,0.36,0.00}{\textbf{#1}}}
\newcommand{\OtherTok}[1]{\textcolor[rgb]{0.56,0.35,0.01}{#1}}
\newcommand{\PreprocessorTok}[1]{\textcolor[rgb]{0.56,0.35,0.01}{\textit{#1}}}
\newcommand{\RegionMarkerTok}[1]{#1}
\newcommand{\SpecialCharTok}[1]{\textcolor[rgb]{0.00,0.00,0.00}{#1}}
\newcommand{\SpecialStringTok}[1]{\textcolor[rgb]{0.31,0.60,0.02}{#1}}
\newcommand{\StringTok}[1]{\textcolor[rgb]{0.31,0.60,0.02}{#1}}
\newcommand{\VariableTok}[1]{\textcolor[rgb]{0.00,0.00,0.00}{#1}}
\newcommand{\VerbatimStringTok}[1]{\textcolor[rgb]{0.31,0.60,0.02}{#1}}
\newcommand{\WarningTok}[1]{\textcolor[rgb]{0.56,0.35,0.01}{\textbf{\textit{#1}}}}
\usepackage{graphicx}
\makeatletter
\def\maxwidth{\ifdim\Gin@nat@width>\linewidth\linewidth\else\Gin@nat@width\fi}
\def\maxheight{\ifdim\Gin@nat@height>\textheight\textheight\else\Gin@nat@height\fi}
\makeatother
% Scale images if necessary, so that they will not overflow the page
% margins by default, and it is still possible to overwrite the defaults
% using explicit options in \includegraphics[width, height, ...]{}
\setkeys{Gin}{width=\maxwidth,height=\maxheight,keepaspectratio}
% Set default figure placement to htbp
\makeatletter
\def\fps@figure{htbp}
\makeatother
\setlength{\emergencystretch}{3em} % prevent overfull lines
\providecommand{\tightlist}{%
  \setlength{\itemsep}{0pt}\setlength{\parskip}{0pt}}
\setcounter{secnumdepth}{-\maxdimen} % remove section numbering
\ifLuaTeX
  \usepackage{selnolig}  % disable illegal ligatures
\fi
\IfFileExists{bookmark.sty}{\usepackage{bookmark}}{\usepackage{hyperref}}
\IfFileExists{xurl.sty}{\usepackage{xurl}}{} % add URL line breaks if available
\urlstyle{same} % disable monospaced font for URLs
\hypersetup{
  pdftitle={Taper\_paper\_with\_timbeR},
  pdfauthor={Ananda},
  hidelinks,
  pdfcreator={LaTeX via pandoc}}

\title{Taper\_paper\_with\_timbeR}
\author{Ananda}
\date{2024-01-26}

\begin{document}
\maketitle

\hypertarget{r-markdown}{%
\subsection{R Markdown}\label{r-markdown}}

This is an R Markdown document. Markdown is a simple formatting syntax
for authoring HTML, PDF, and MS Word documents. For more details on
using R Markdown see \url{http://rmarkdown.rstudio.com}.

When you click the \textbf{Knit} button a document will be generated
that includes both content as well as the output of any embedded R code
chunks within the document. You can embed an R code chunk like this:

\begin{Shaded}
\begin{Highlighting}[]
\FunctionTok{summary}\NormalTok{(cars)}
\end{Highlighting}
\end{Shaded}

\begin{verbatim}
##      speed           dist       
##  Min.   : 4.0   Min.   :  2.00  
##  1st Qu.:12.0   1st Qu.: 26.00  
##  Median :15.0   Median : 36.00  
##  Mean   :15.4   Mean   : 42.98  
##  3rd Qu.:19.0   3rd Qu.: 56.00  
##  Max.   :25.0   Max.   :120.00
\end{verbatim}

\hypertarget{including-plots}{%
\subsection{Including Plots}\label{including-plots}}

You can also embed plots, for example:

\includegraphics{Taper_v1_files/figure-latex/pressure-1.pdf}

Note that the \texttt{echo\ =\ FALSE} parameter was added to the code
chunk to prevent printing of the R code that generated the plot.

\hypertarget{developing-taper-equations-for-sal}{%
\section{Developing taper equations for
Sal}\label{developing-taper-equations-for-sal}}

\hypertarget{script-developed-by-ananda-khadka-last-modified-4th-september-2022}{%
\section{Script developed by Ananda Khadka, last modified 4th September,
2022}\label{script-developed-by-ananda-khadka-last-modified-4th-september-2022}}

\hypertarget{citation}{%
\subsection{Citation}\label{citation}}

citation(``rForest'')

\hypertarget{data}{%
\paragraph{data}\label{data}}

\hypertarget{dat---taper_data_final}{%
\section{dat \textless-
taper\_data\_final}\label{dat---taper_data_final}}

\hypertarget{headdat}{%
\section{head(dat)}\label{headdat}}

\hypertarget{dat-is-ready-for-analysis-data-all}{%
\section{dat is ready for analysis--- data
all}\label{dat-is-ready-for-analysis-data-all}}

\hypertarget{copy-of-dat}{%
\section{copy of dat}\label{copy-of-dat}}

DAT\textless- dat

\hypertarget{new-dataset-removing-the-outliers}{%
\section{new dataset removing the
outliers}\label{new-dataset-removing-the-outliers}}

\hypertarget{view-outliers}{%
\section{view outliers}\label{view-outliers}}

plot(dat\(Hx, dat\)Dx)

\hypertarget{lets-say-trees-with-dx-more-than-120-cm-are-outliers}{%
\section{lets say, trees with Dx more than 120 cm are
outliers}\label{lets-say-trees-with-dx-more-than-120-cm-are-outliers}}

\hypertarget{get-new-dataset-dat120-having-safety-copy-dat}{%
\section{get new dataset ``dat120'', having safety copy
DAT}\label{get-new-dataset-dat120-having-safety-copy-dat}}

dat90\textless-subset(DAT, DAT\$dbh\textless90)

dat90\textless-subset(dat90, dat90\$Dx\textless101)

plot(dat90\(Hx, dat90\)Dx)

\hypertarget{for-convenience-rename-dat120-as-dat}{%
\section{for convenience, rename ``dat120'' as
``dat''}\label{for-convenience-rename-dat120-as-dat}}

dat\textless-dat90

\hypertarget{count-number-of-tree-id}{%
\section{count number of tree id}\label{count-number-of-tree-id}}

length(unique(dat\$Id))

with(dat, plot(Hx, Dx, col=Id) )

\hypertarget{datid-datid}{%
\section{\texorpdfstring{dat\(Id<-dat\)id}{datId\textless-datid}}\label{datid-datid}}

\hypertarget{dathx-dathx}{%
\section{\texorpdfstring{dat\(Hx<-dat\)hx}{datHx\textless-dathx}}\label{dathx-dathx}}

\hypertarget{datdx-datdx}{%
\section{\texorpdfstring{dat\(Dx<-dat\)dx}{datDx\textless-datdx}}\label{datdx-datdx}}

\hypertarget{datht-dath_total}{%
\section{\texorpdfstring{dat\(Ht<-dat\)h\_total}{datHt\textless-dath\_total}}\label{datht-dath_total}}

\hypertarget{observation}{%
\paragraph{Observation}\label{observation}}

\hypertarget{view-observation-of-h-and-d-for-all-trees}{%
\section{view observation of H and D for all
trees}\label{view-observation-of-h-and-d-for-all-trees}}

with(dat, plot(dat\(Hx, dat\)Dx)) with(dat, plot(Hx, Dx, col=Id, pch=20,
cex=0.5))

with(dat, plot(Hx, Dx, col=Id, pch=20, cex=0.7, xlab=``Height from the
tree base (m)'', ylab=``Stem diameter (cm)'' ))\\
\# install.packages(``ggplot2'') \# install.packages(``readr'') \#
install.packages(``dplyr'') library(ggplot2) library(readr)
library(dplyr)

ggplot(data = dat, aes(x = dat\$Hx))

ggplot(data = dat, aes(x = dat\$Hx)) + geom\_histogram()

\hypertarget{dbh1---datdbh-summarizemean_price-meanprice-price_stats-ggplotdata-dat-aesx-dathx}{%
\section{\texorpdfstring{dbh1 \textless-
dat\(dbh |> # summarize(mean_price = mean(price)) # price_stats # # ggplot(data = dat, aes(x = dat\)Hx))
+}{dbh1 \textless- datdbh \textbar\textgreater{} \# summarize(mean\_price = mean(price)) \# price\_stats \# \# ggplot(data = dat, aes(x = datHx)) +}}\label{dbh1---datdbh-summarizemean_price-meanprice-price_stats-ggplotdata-dat-aesx-dathx}}

\hypertarget{geom_histogram}{%
\section{geom\_histogram()+}\label{geom_histogram}}

\hypertarget{geom_vlineaesxintercept-datdbh-datdbh-color-red-linewidth-2}{%
\section{\texorpdfstring{geom\_vline(aes(xintercept =
dat\(dbh), # dat\)dbh, color = ``red'', linewidth =
2)}{geom\_vline(aes(xintercept = datdbh), \# datdbh, color = ``red'', linewidth = 2)}}\label{geom_vlineaesxintercept-datdbh-datdbh-color-red-linewidth-2}}

\hypertarget{plot-only-one-tree-data}{%
\section{plot only one tree data}\label{plot-only-one-tree-data}}

with(dat{[} dat\(Id ==15,], plot(Hx, Dx)) with(dat[ dat\)Id ==53,{]},
plot(Hx, Dx)) with(dat{[}
dat\(Id ==66,], plot(Hx, Dx)) with(dat[ dat\)Id ==78,{]}, plot(Hx, Dx))

\hypertarget{parmfrowc22}{%
\paragraph{par(mfrow=c(2,2))}\label{parmfrowc22}}

\hypertarget{parmfrowc11}{%
\paragraph{par(mfrow=c(1,1))}\label{parmfrowc11}}

\#plot with ggplot2 \# ggplot(dat, aes(Hx, Dx)) + geom\_point() +
geom\_smooth()\#stat\_quantile() \#\\
\# ggplot(dat, aes(Hx, Dx)) + geom\_point() + geom\_smooth()

\hypertarget{data-1}{%
\section{data}\label{data-1}}

dat \# prepare the data (could be defined in the function directly) Id =
dat{[},``Id''{]}

\hypertarget{main-calculation-starts-from-here}{%
\paragraph{Main calculation starts from
here}\label{main-calculation-starts-from-here}}

\hypertarget{derive-x-and-y}{%
\paragraph{Derive x and y}\label{derive-x-and-y}}

x = dat{[},``Hx''{]}/dat{[},``Ht''{]} \#calculate relative heights

\hypertarget{add-a-column-x-in-dat}{%
\section{add a column ``x'' in dat}\label{add-a-column-x-in-dat}}

dat\$x\textless-x

y = dat{[},``Dx''{]} \# upper stem diameters at different heights

\hypertarget{add-a-column-y-in-dat}{%
\section{add a column ``y'' in dat}\label{add-a-column-y-in-dat}}

dat\$y\textless-y

\hypertarget{library}{%
\paragraph{library}\label{library}}

library(tidyverse) library(caret) library(ggpmisc) library(splines)
library(ggplot2) library(ModelMetrics) library(rForest) library(rgl)
library(devtools) library(broom) library(dplyr) library(cowplot)
library(ggpubr)

\hypertarget{observe-dbh-distribution}{%
\section{observe dbh distribution}\label{observe-dbh-distribution}}

hist(DAT\(dbh) hist(unique(dat\)dbh), main = ``DBH distribution of
sample trees'', xlab = ``DBH in cm'', ylab = ``Number of sample trees'')

\hypertarget{try-model-1-for-dbh-up-to-50-cm}{%
\section{try model 1 for dbh up to 50
cm}\label{try-model-1-for-dbh-up-to-50-cm}}

\hypertarget{try-model-2-for-dbh-50.1-to-80-cm}{%
\section{try model 2 for dbh 50.1 to 80
cm}\label{try-model-2-for-dbh-50.1-to-80-cm}}

\hypertarget{try-model-3-for-dbh-more-than-80.1-cm}{%
\section{try model 3 for dbh more than 80.1
cm}\label{try-model-3-for-dbh-more-than-80.1-cm}}

\hypertarget{subsetting-the-data-into-3-dbh-classes}{%
\section{subsetting the data into 3 dbh
classes}\label{subsetting-the-data-into-3-dbh-classes}}

dat1\textless-subset(dat,dat{[},19{]}\textless55.1)
dat2\textless-subset(dat,(dat{[},19{]}\textgreater55.0 \&
dat{[},19{]}\textless70.1))
dat3\textless-subset(dat,dat{[},19{]}\textgreater70.0)

hist(unique(dat\$dbh), main = ``All DBH size'', xlab = ``DBH in cm'',
ylab = ``Number of sample trees'')

hist(unique(dat1\$dbh), main = ``DBH class \textless{} 55 cm'', xlab =
``DBH in cm'', ylab = ``Number of sample trees'')

hist(unique(dat2\$dbh), main = ``DBH class 55 - 70 cm'', xlab = ``DBH in
cm'', ylab = ``Number of sample trees'')

hist(unique(dat3\$dbh), main = ``DBH class \textgreater{} 70 cm'', xlab
= ``DBH in cm'', ylab = ``Number of sample trees'')

\hypertarget{see-dbh-in-different-subsets}{%
\section{see dbh in different
subsets}\label{see-dbh-in-different-subsets}}

\hypertarget{histdat1dbh-histdat2dbh}{%
\section{\texorpdfstring{hist(dat1\(dbh) # hist(dat2\)dbh)}{hist(dat1dbh) \# hist(dat2dbh)}}\label{histdat1dbh-histdat2dbh}}

\hypertarget{histdat3dbh}{%
\section{hist(dat3\$dbh)}\label{histdat3dbh}}

hist(dat\(dbh, xlab = "DBH (All size in cm)", main = "Number of trees by DBH size") hist(dat1\)dbh,
xlab = ``DBH (upto 55 cm)'', main = ``Number of trees by DBH size'')
hist(dat2\(dbh, xlab = "DBH (55-70 cm)", main = "Number of trees by DBH size") hist(dat3\)dbh,
xlab = ``DBH (more than 70 cm)'', main = ``Number of trees by DBH
size'')

\hypertarget{linear-regression}{%
\paragraph{Linear regression}\label{linear-regression}}

\hypertarget{the-standard-linear-regression-model-equation-can-be-written-as}{%
\section{The standard linear regression model equation can be written
as:}\label{the-standard-linear-regression-model-equation-can-be-written-as}}

\hypertarget{upper_stem_dia-a-bh.}{%
\section{upper\_stem\_dia = a + b*h.}\label{upper_stem_dia-a-bh.}}

\hypertarget{compute-linear-regression-modelr}{%
\section{Compute linear regression model:```\{r
\}}\label{compute-linear-regression-modelr}}

\hypertarget{build-the-model}{%
\section{Build the model}\label{build-the-model}}

\hypertarget{linear-model-for-dbh-up-to-40-cm}{%
\section{linear model for dbh up to 40
cm}\label{linear-model-for-dbh-up-to-40-cm}}

lm1 \textless- lm(y \textasciitilde{} x, data = dat1) summary(lm1)

par(mfrow=c(1,1)) plot(dat1\(x, dat1\)y, col=``blue'', xlab = ``Relative
heights'', ylab = ``Upper stem diameters (cm)'', main = ``Observation:
upper stem diameter by relative heights'', cex=0.5, cex.lab=0.7,
cex.main=0.7)

par(mfrow=c(2,2))

plot(lm1 \textless- lm(y \textasciitilde{} x, data = dat1), cex=0.3,
cex.lab=0.8, cex.main=0.5)

ggplot(dat1, aes(x, y) ) + geom\_point() + stat\_smooth(method = lm,
formula = y \textasciitilde{} x)+ stat\_poly\_eq(aes(label =
paste0(``atop('', ..eq.label.., ``,'', ..rr.label.., ``)'')), formula =
y \textasciitilde{} x, parse = TRUE) + theme\_bw(base\_size = 12)

\hypertarget{make-predictions}{%
\section{Make predictions}\label{make-predictions}}

\hypertarget{librarymodelmetrics}{%
\section{library(ModelMetrics)}\label{librarymodelmetrics}}

dat1\(pred_lm1 <- lm1 %>% predict(dat1) # Model performance rmse_lm1<- rmse(dat1
\)pred\_lm1, dat1\(Dx) rmse_lm1 mean(dat1\)dbh)

\hypertarget{linear-model-for-dbh-40-70-cm}{%
\section{linear model for dbh 40-70
cm}\label{linear-model-for-dbh-40-70-cm}}

lm2 \textless- lm(y \textasciitilde{} x, data = dat2) summary(lm2)

par(mfrow=c(1,1))

plot(dat2\(x, dat2\)y, col=``blue'', xlab = ``Relative heights'', ylab =
``Upper stem diameters (cm)'', main = ``Observation: upper stem diameter
by relative heights'', cex=0.5, cex.lab=0.7, cex.main=0.7)

par(mfrow=c(2,2)) plot(lm2 \textless- lm(y \textasciitilde{} x, data =
dat2), cex=0.3, cex.lab=0.8, cex.main=0.5)

par(mfrow=c(1,1))

ggplot(dat2, aes(x, y) ) + geom\_point() + stat\_smooth(method = lm,
formula = y \textasciitilde{} x)+ stat\_poly\_eq(aes(label =
paste0(``atop('', ..eq.label.., ``,'', ..rr.label.., ``)'')), formula =
y \textasciitilde{} x, parse = TRUE) + theme\_bw(base\_size = 12)

\hypertarget{make-predictions-1}{%
\section{Make predictions}\label{make-predictions-1}}

\hypertarget{librarymodelmetrics-1}{%
\section{library(ModelMetrics)}\label{librarymodelmetrics-1}}

dat2\(pred_lm2 <- lm2 %>% predict(dat2) # Model performance rmse_lm2<- rmse(dat2
\)pred\_lm2, dat2\(Dx) rmse_lm2 mean(dat2\)dbh)

\hypertarget{linear-model-for-dbh-70-cm}{%
\section{linear model for dbh \textgreater{} 70
cm}\label{linear-model-for-dbh-70-cm}}

lm3 \textless- lm(y \textasciitilde{} x, data = dat3) summary(lm3)

par(mfrow=c(1,1))

plot(dat3\(x, dat3\)y, col=``blue'', xlab = ``Relative heights'', ylab =
``Upper stem diameters (cm)'', main = ``Observation: upper stem diameter
by relative heights'', cex=0.5, cex.lab=0.7, cex.main=0.7)

par(mfrow=c(2,2)) plot(lm3 \textless- lm(y \textasciitilde{} x, data =
dat3), cex=0.3, cex.lab=0.8, cex.main=0.5)

par(mfrow=c(1,1))\\
ggplot(dat3, aes(x, y) ) + geom\_point() + stat\_smooth(method = lm,
formula = y \textasciitilde{} x)+ stat\_poly\_eq(aes(label =
paste0(``atop('', ..eq.label.., ``,'', ..rr.label.., ``)'')), formula =
y \textasciitilde{} x, parse = TRUE) + theme\_bw(base\_size = 12)

\hypertarget{make-predictions-2}{%
\section{Make predictions}\label{make-predictions-2}}

\hypertarget{librarymodelmetrics-2}{%
\section{library(ModelMetrics)}\label{librarymodelmetrics-2}}

dat3\(pred_lm3 <- lm3 %>% predict(dat3) # Model performance rmse_lm3<- rmse(dat3
\)pred\_lm3, dat3\(Dx) rmse_lm3 mean(dat3\)dbh)

\#\#\#\#Polynominal\#\#\#\#

\hypertarget{polynomial-model-for-dbh-55-cm}{%
\section{Polynomial model for dbh \textless55
cm}\label{polynomial-model-for-dbh-55-cm}}

pm1\textless-lm(y \textasciitilde{} poly(x, 2, raw=TRUE), data = dat1)
summary(pm1)

par(mfrow=c(2,2))

plot(pm1 \textless- lm(y \textasciitilde{} poly(x, 2, raw = TRUE), data
= dat1), cex=0.3, cex.lab=0.8, cex.main=0.5)

par(mfrow=c(1,1))

ggplot(dat1, aes(x, y) ) + geom\_point() + stat\_smooth(method = lm,
formula = y \textasciitilde{} poly(x, 2, raw = TRUE))+
stat\_poly\_eq(aes(label = paste0(``atop('', ..eq.label.., ``,'',
..rr.label.., ``)'')), formula = y \textasciitilde{} poly(x, 2, raw =
TRUE), data = dat1, parse = TRUE) + theme\_bw(base\_size = 12)

\hypertarget{make-predictions-3}{%
\section{Make predictions}\label{make-predictions-3}}

\hypertarget{librarymodelmetrics-3}{%
\section{library(ModelMetrics)}\label{librarymodelmetrics-3}}

dat1\(pred_pm1 <- pm1 %>% predict(dat1) # Model performance rmse_pm1<- rmse(dat1
\)pred\_pm1, dat1\(Dx) rmse_pm1 mean(dat1\)dbh)

\hypertarget{polynomial-model-for-dbh-55-70-cm}{%
\section{Polynomial model for dbh 55-70
cm}\label{polynomial-model-for-dbh-55-70-cm}}

pm2\textless-lm(y \textasciitilde{} poly(x, 2, raw=TRUE), data = dat2)

par(mfrow=c(2,2))

plot(pm2 \textless- lm(y \textasciitilde{} poly(x, 2, raw = TRUE), data
= dat2), cex=0.3, cex.lab=0.8, cex.main=0.5)

ggplot(dat2, aes(x, y) ) + geom\_point() + stat\_smooth(method = lm,
formula = y \textasciitilde{} poly(x, 2, raw = TRUE))+
stat\_poly\_eq(aes(label = paste0(``atop('', ..eq.label.., ``,'',
..rr.label.., ``)'')), formula = y \textasciitilde{} poly(x, 2, raw =
TRUE), parse = TRUE) + theme\_bw(base\_size = 12)

\hypertarget{make-predictions-4}{%
\section{Make predictions}\label{make-predictions-4}}

\hypertarget{librarymodelmetrics-4}{%
\section{library(ModelMetrics)}\label{librarymodelmetrics-4}}

dat2\(pred_pm2 <- pm2 %>% predict(dat2) # Model performance rmse_pm2<- rmse(dat2
\)pred\_pm2, dat2\$Dx) rmse\_pm2

\hypertarget{polynomial-model-for-dbh-70-cm}{%
\section{Polynomial model for dbh \textgreater{} 70
cm}\label{polynomial-model-for-dbh-70-cm}}

pm3\textless-lm(y \textasciitilde{} poly(x, 2, raw=TRUE), data = dat3)

par(mfrow=c(2,2))

plot(pm3 \textless- lm(y \textasciitilde{} poly(x, 2, raw = TRUE), data
= dat3), cex=0.3, cex.lab=0.8, cex.main=0.5)

ggplot(dat3, aes(x, y) ) + geom\_point() + stat\_smooth(method = lm,
formula = y \textasciitilde{} poly(x, 2, raw = TRUE))+
stat\_poly\_eq(aes(label = paste0(``atop('', ..eq.label.., ``,'',
..rr.label.., ``)'')), formula = y \textasciitilde{} poly(x, 2, raw =
TRUE), parse = TRUE) + theme\_bw(base\_size = 12)

\hypertarget{make-predictions-5}{%
\section{Make predictions}\label{make-predictions-5}}

\hypertarget{librarymodelmetrics-5}{%
\section{library(ModelMetrics)}\label{librarymodelmetrics-5}}

dat3\(pred_pm3 <- pm3 %>% predict(dat3) # Model performance rmse_pm3<- rmse(dat3
\)pred\_pm3, dat3\$Dx) rmse\_pm3

\#\#\#\#Spline\#\#\#\# \# library(splines) \# Build the model

\hypertarget{bspline-regression-model-for-dbh55cm}{%
\section{Bspline regression model for
dbh\textless55cm}\label{bspline-regression-model-for-dbh55cm}}

knots \textless-
quantile(dat1\(x, p = c(0.25, 0.5, 0.75)) sm1 <- lm (dat1\)y
\textasciitilde{} bs(dat1\$x, knots = knots), data = dat1) summary(sm1)

\hypertarget{plot}{%
\section{plot}\label{plot}}

par(mfrow=c(1,1)) ggplot(dat1, aes(x, y) ) + geom\_point() +
stat\_smooth(method = lm, formula = y \textasciitilde{} splines::bs(x,
df = 3))+ stat\_poly\_eq(aes(label = paste0(``atop('', ..eq.label..,
``,'', ..rr.label.., ``)'')), formula = y \textasciitilde{}
splines::bs(x, df = 3), parse = TRUE) + theme\_bw(base\_size = 12)

par(mfrow=c(2,2)) plot(sm1 \textless- lm (dat1\(y ~ bs(dat1\)x, knots =
knots), data = dat1), cex=0.3, cex.lab=0.8, cex.main=0.5)

\hypertarget{make-predictions-6}{%
\section{Make predictions}\label{make-predictions-6}}

\hypertarget{librarymodelmetrics-6}{%
\section{library(ModelMetrics)}\label{librarymodelmetrics-6}}

dat1\$pred\_sm1 \textless- sm1 \%\textgreater\% predict(dat1)

\hypertarget{model-performance}{%
\section{Model performance}\label{model-performance}}

rmse\_sm1\textless- rmse(dat1\(pred_sm1, dat1\)Dx) rmse\_sm1

\hypertarget{bspline-regression-model-for-dbh-55-70-cm}{%
\section{Bspline regression model for dbh 55-70
cm}\label{bspline-regression-model-for-dbh-55-70-cm}}

knots \textless-
quantile(dat2\(x, p = c(0.25, 0.5, 0.75)) sm2 <- lm (dat2\)y
\textasciitilde{} bs(dat2\$x, knots = knots), data = dat2) summary(sm2)

par(mfrow=c(1,1)) ggplot(dat2, aes(x, y) ) + geom\_point() +
stat\_smooth(method = lm, formula = y \textasciitilde{} splines::bs(x,
df = 3))+ stat\_poly\_eq(aes(label = paste0(``atop('', ..eq.label..,
``,'', ..rr.label.., ``)'')), formula = y \textasciitilde{}
splines::bs(x, df = 3), parse = TRUE) + theme\_bw(base\_size = 12)

par(mfrow=c(2,2)) plot(sm2 \textless- lm (dat2\(y ~ bs(dat2\)x, knots =
knots), data = dat2), cex=0.3, cex.lab=0.8, cex.main=0.5)

\hypertarget{make-predictions-7}{%
\section{Make predictions}\label{make-predictions-7}}

\hypertarget{librarymodelmetrics-7}{%
\section{library(ModelMetrics)}\label{librarymodelmetrics-7}}

dat2\$pred\_sm2 \textless- sm2 \%\textgreater\% predict(dat2)

\hypertarget{model-performance-1}{%
\section{Model performance}\label{model-performance-1}}

rmse\_sm2\textless- rmse(dat2\(pred_sm2, dat2\)Dx) rmse\_sm2

\hypertarget{bspline-regression-model-for-dbh-70-cm}{%
\section{Bspline regression model for dbh \textgreater{} 70
cm}\label{bspline-regression-model-for-dbh-70-cm}}

knots \textless-
quantile(dat3\(x, p = c(0.25, 0.5, 0.75)) sm3 <- lm (dat3\)y
\textasciitilde{} bs(dat3\$x, knots = knots), data = dat3) summary(sm3)

par(mfrow=c(1,1)) ggplot(dat3, aes(x, y) ) + geom\_point() +
stat\_smooth(method = lm, formula = y \textasciitilde{} splines::bs(x,
df = 3))+ stat\_poly\_eq(aes(label = paste0(``atop('', ..eq.label..,
``,'', ..rr.label.., ``)'')), formula = y \textasciitilde{}
splines::bs(x, df = 3), parse = TRUE) + theme\_bw(base\_size = 12)

par(mfrow=c(2,2)) plot(sm3 \textless- lm (dat3\(y ~ bs(dat3\)x, knots =
knots), data = dat3), cex=0.3, cex.lab=0.8, cex.main=0.5)

\hypertarget{make-predictions-8}{%
\section{Make predictions}\label{make-predictions-8}}

\hypertarget{librarymodelmetrics-8}{%
\section{library(ModelMetrics)}\label{librarymodelmetrics-8}}

dat3\$pred\_sm3 \textless- sm3 \%\textgreater\% predict(dat3)

\hypertarget{model-performance-2}{%
\section{Model performance}\label{model-performance-2}}

rmse\_sm3\textless- rmse(dat3\(pred_sm3, dat3\)Dx) rmse\_sm3

\hypertarget{bspline-regression-model-for-all-dbh-class}{%
\section{Bspline regression model for all dbh
class}\label{bspline-regression-model-for-all-dbh-class}}

knots \textless-
quantile(dat\(x, p = c(0.25, 0.5, 0.75)) sm <- lm (dat\)y
\textasciitilde{} bs(dat\$x, knots = knots), data = dat) summary(sm)

par(mfrow=c(1,1)) ggplot(dat, aes(x, y) ) + geom\_point() +
stat\_smooth(method = lm, formula = y \textasciitilde{} splines::bs(x,
df = 3))+ stat\_poly\_eq(aes(label = paste0(``atop('', ..eq.label..,
``,'', ..rr.label.., ``)'')), formula = y \textasciitilde{}
splines::bs(x, df = 3), parse = TRUE) + theme\_bw(base\_size = 12)

par(mfrow=c(2,2)) plot(sm \textless- lm (dat\(y ~ bs(dat\)x, knots =
knots), data = dat), cex=0.3, cex.lab=0.8, cex.main=0.5)

\hypertarget{make-predictions-9}{%
\section{Make predictions}\label{make-predictions-9}}

\hypertarget{librarymodelmetrics-9}{%
\section{library(ModelMetrics)}\label{librarymodelmetrics-9}}

dat\$pred\_sm \textless- sm \%\textgreater\% predict(dat)

\hypertarget{model-performance-3}{%
\section{Model performance}\label{model-performance-3}}

rmse\_sm\textless- rmse(dat\(pred_sm, dat\)Dx) rmse\_sm

\hypertarget{plots-in-1-figure}{%
\subsection{4 plots in 1 figure}\label{plots-in-1-figure}}

\hypertarget{since-updated-rstudio-some-packages-needed-to-remove-and-install-again}{%
\section{since updated Rstudio, some packages needed to remove and
install
again}\label{since-updated-rstudio-some-packages-needed-to-remove-and-install-again}}

\hypertarget{install.packagesggplot2}{%
\section{install.packages(``ggplot2'')}\label{install.packagesggplot2}}

fig1\textless-ggplot(dat1, aes(x, y) ) + geom\_point() +
stat\_smooth(method = lm, formula = y \textasciitilde{} splines::bs(x,
df = 3))+ stat\_poly\_eq(aes(label = paste0(``atop('', ..eq.label..,
``,'', ..rr.label.., ``)'')), formula = y \textasciitilde{}
splines::bs(x, df = 3), parse = TRUE) + theme\_bw(base\_size = 12)+
xlim(c(0,1.0))+ ylim(c(0,110))

fig1

fig2\textless-ggplot(dat2, aes(x, y) ) + geom\_point() +
stat\_smooth(method = lm, formula = y \textasciitilde{} splines::bs(x,
df = 3))+ stat\_poly\_eq(aes(label = paste0(``atop('', ..eq.label..,
``,'', ..rr.label.., ``)'')), formula = y \textasciitilde{}
splines::bs(x, df = 3), parse = TRUE) + theme\_bw(base\_size = 12)+
xlim(c(0,1.0))+ ylim(c(0,110))

fig2

fig3\textless- ggplot(dat3, aes(x, y) ) + geom\_point() +
stat\_smooth(method = lm, formula = y \textasciitilde{} splines::bs(x,
df = 3))+ stat\_poly\_eq(aes(label = paste0(``atop('', ..eq.label..,
``,'', ..rr.label.., ``)'')), formula = y \textasciitilde{}
splines::bs(x, df = 3), parse = TRUE) + theme\_bw(base\_size = 12)+
xlim(c(0,1.0))+ ylim(c(0,110))

fig3

fig4\textless-ggplot(dat, aes(x, y) ) + geom\_point() +
stat\_smooth(method = lm, formula = y \textasciitilde{} splines::bs(x,
df = 3))+ stat\_poly\_eq(aes(label = paste0(``atop('', ..eq.label..,
``,'', ..rr.label.., ``)'')), formula = y \textasciitilde{}
splines::bs(x, df = 3), parse = TRUE) + theme\_bw(base\_size = 12) +
xlim(c(0,1.0))+ ylim(c(0,110))

fig4

\hypertarget{in-a-single-figure}{%
\section{in a single figure}\label{in-a-single-figure}}

\hypertarget{install.packageggpubr}{%
\section{install.package(ggpubr)}\label{install.packageggpubr}}

\hypertarget{library-ggpubr}{%
\section{library (ggpubr)}\label{library-ggpubr}}

\hypertarget{install.packagecowplot}{%
\section{install.package(cowplot)}\label{install.packagecowplot}}

\hypertarget{library-cowplot}{%
\section{library (cowplot)}\label{library-cowplot}}

plot\_grid(fig1, fig2, fig3, fig4, labels = c(``Model for DBH up to 50
cm'', ``Model for DBH class 55 - 70 cm'', ``Model for DBH greater than
70 cm'', ``Model for all DBH class''), label\_size = 10, hjust = -1.75,
vjust = 2.0, \# align = ``h'', axis = ``tb'', ncol = 2, nrow = 2)

\hypertarget{ggarrangefig1-fig2-fig3-fig4}{%
\section{ggarrange(fig1, fig2, fig3,
fig4,}\label{ggarrangefig1-fig2-fig3-fig4}}

\hypertarget{labels-cfigure-5-model-for-dbh-up-to-50-cm}{%
\section{labels = c(``Figure 5: Model for DBH up to 50
cm'',}\label{labels-cfigure-5-model-for-dbh-up-to-50-cm}}

\hypertarget{figure-6-model-for-dbh-class-55---70-cm}{%
\section{``Figure 6: Model for DBH class 55 - 70
cm'',}\label{figure-6-model-for-dbh-class-55---70-cm}}

\hypertarget{figure-7-model-for-dbh-greater-than-70-cm}{%
\section{``Figure 7: Model for DBH greater than 70
cm'',}\label{figure-7-model-for-dbh-greater-than-70-cm}}

\hypertarget{figure-8-model-for-all-dbh-class}{%
\section{``Figure 8: Model for all DBH
class''),}\label{figure-8-model-for-all-dbh-class}}

\hypertarget{ncol-2-nrow-2}{%
\section{ncol = 2, nrow = 2)}\label{ncol-2-nrow-2}}

\hypertarget{fifth-degree-polynomial-taper-model}{%
\paragraph{fifth-degree polynomial taper
model}\label{fifth-degree-polynomial-taper-model}}

\#carlos-alberto-silva/rForest \# rForest: An R Package for Forest
Inventory and Analysis \# The rForest package provides functions to \#
i) Fit a fifth-degree polynomial taper model, \# ii) plot tree stems in
2-D and 3-D, \# iii) plot taper models in 3-D.

\hypertarget{installation}{%
\section{Installation}\label{installation}}

\#The development version: \# library(devtools) \#
devtools::install\_github(``carlos-alberto-silva/rForest'')

\#The CRAN version: \# install.packages(``rForest'')

\hypertarget{d-visualization-of-tree-stems}{%
\paragraph{2-D visualization of tree
stems}\label{d-visualization-of-tree-stems}}

\#Loading rForest and rgl libraries \# library(rForest) \# library(rgl)
\# library(devtools) \# Importing forest inventory data \#
data(ForestInv01) names(dat)

\hypertarget{subsetting-tree-1}{%
\section{Subsetting Tree 1}\label{subsetting-tree-1}}

tree1\textless-subset(dat,dat{[},3{]}==1)

hi\textless-tree1\(Hx di<-tree1\)Dx

\hypertarget{plotting-stem-2d}{%
\section{Plotting stem 2d}\label{plotting-stem-2d}}

plotStem2d(hi,di, col=``\#654321'')

\hypertarget{d-visualization-of-tree-stems-1}{%
\paragraph{3-D visualization of tree
stems}\label{d-visualization-of-tree-stems-1}}

plotStem3d(hi,di,alpha=1,col=``\#654321'') box3d()

\hypertarget{fitting-a-fifth-degree-polynomial-taper-model}{%
\paragraph{Fitting a fifth-degree polynomial taper
model}\label{fitting-a-fifth-degree-polynomial-taper-model}}

\hypertarget{setting-model-parameters-dbh-and-ht-for-all-dbh-class}{%
\section{setting model parameters dbh and ht for ALL DBH
CLASS}\label{setting-model-parameters-dbh-and-ht-for-all-dbh-class}}

names(dat)

hi\textless-dat{[},20{]} di\textless-dat{[},21{]}
ht\textless-dat{[},18{]} dbh\textless-dat{[},19{]}

\hypertarget{fitting-the-fifth-degree-polynomial-taper-model}{%
\section{fitting the fifth-degree polynomial taper
model}\label{fitting-the-fifth-degree-polynomial-taper-model}}

fit \textless- poly5Model(dbh,ht,di,hi, plotxy=TRUE) grid()

par(mfcol=c(2,2)) plot(fit)

summary(fit)

\hypertarget{new-way-to-run-polynomial-5th-degree-taper-model}{%
\paragraph{new way to run polynomial 5th degree taper
model}\label{new-way-to-run-polynomial-5th-degree-taper-model}}

library(ggplot2)

dat \textless- dat \%\textgreater\% mutate(did = Dx/dbh, hih = Hx/Ht)

ggplot(dat, aes(x = hih, y = did, group = tree\_id))+ geom\_point()+
labs(x = `hi / h', y = `di / dbh') \# run this if error like: ``Error in
.Call.graphics(C\_palette2, .Call(C\_palette2, NULL)) : \# invalid
graphics state'' happens dev.off()

poli5 \textless-
lm(did\textasciitilde hih+I(hih\textsuperscript{2)+I(hih}3)+I(hih\textsuperscript{4)+I(hih}5),dat)
summary(poli5)

dat \textless- dat \%\textgreater\% mutate(Dx\_poli =
predict(poli5)*dbh)

poli\_rmse \textless- dat \%\textgreater\% summarise(RMSE =
sqrt(sum((Dx\_poli-Dx)\^{}2)/mean(Dx\_poli))) \%\textgreater\%
pull(RMSE) \%\textgreater\% round(2)

ggplot(dat,aes(x=hih))+ geom\_point(aes(y =
(Dx\_poli-Dx)/Dx\_poli*100))+ geom\_hline(aes(yintercept = 0))+
scale\_y\_continuous(limits=c(-60,60), breaks = seq(-100,100,20))+
scale\_x\_continuous(limits=c(0,1))+ labs(x = `hi / h', y = `Residuals
(\%)', title = `5th degree polynomial taper function (Schöepfer, 1966)',
subtitle = `Dispersion of residuals along the stem', caption =
paste0(`Root Mean Squared Error =', poli\_rmse,`\%'))+
theme(plot.title.position = `plot')

\hypertarget{timber-package}{%
\section{timbeR package}\label{timber-package}}

\hypertarget{we-will-perform-a-regression-analysis-on-the-tree_scaling-dataset-using-the}{%
\section{We will perform a regression analysis on the tree\_scaling
dataset, using
the}\label{we-will-perform-a-regression-analysis-on-the-tree_scaling-dataset-using-the}}

\hypertarget{aforementioned-models.-the-data-can-be-accessed-by-importing-the-timber-package.}{%
\section{aforementioned models. The data can be accessed by importing
the timbeR
package.}\label{aforementioned-models.-the-data-can-be-accessed-by-importing-the-timber-package.}}

\hypertarget{install.packagesdevtools}{%
\section{install.packages(``devtools'')}\label{install.packagesdevtools}}

options(download.file.method = ``libcurl'')
devtools::install\_github(`sergiocostafh/timbeR')

library(dplyr) library(timbeR)

glimpse(tree\_scaling)

\hypertarget{as-we-can-see-there-are-five-columns-in-the-dataset-that-refer-to-the-tree-id}{%
\section{As we can see, there are five columns in the dataset that refer
to the tree
id}\label{as-we-can-see-there-are-five-columns-in-the-dataset-that-refer-to-the-tree-id}}

\hypertarget{tree_id-diameter-at-breast-height-dbh-tree-total-height-h-height-at}{%
\section{(tree\_id), diameter at breast height (dbh), tree total height
(h), height
at}\label{tree_id-diameter-at-breast-height-dbh-tree-total-height-h-height-at}}

\hypertarget{section-i-hi-and-diameter-at-hi-height-di.}{%
\section{section i (hi) and diameter at hi height
(di).}\label{section-i-hi-and-diameter-at-hi-height-di.}}

\hypertarget{a-common-way-to-visualize-the-stem-profile-from-collected-data-is-to-plot-the}{%
\section{A common way to visualize the stem profile from collected data
is to plot
the}\label{a-common-way-to-visualize-the-stem-profile-from-collected-data-is-to-plot-the}}

\hypertarget{relationship-between-relative-diameters-and-relative-heights}{%
\section{relationship between relative diameters and relative
heights}\label{relationship-between-relative-diameters-and-relative-heights}}

\hypertarget{di-dbh-vs-hi-ht-as-follows.}{%
\section{(di / dbh vs hi / ht), as
follows.}\label{di-dbh-vs-hi-ht-as-follows.}}

library(ggplot2)

\hypertarget{trying-in-package-dataset-tree_scaling}{%
\section{trying in package dataset
tree\_scaling}\label{trying-in-package-dataset-tree_scaling}}

tree\_scaling \textless- tree\_scaling \%\textgreater\% mutate(did =
di/dbh, hih = hi/h)

ggplot(tree\_scaling, aes(x = hih, y = did, group = tree\_id))+
geom\_point()+ labs(x = `hi / h', y = `di / dbh')

\hypertarget{now-that-we-understand-the-dataset-we-can-start-the-regression-analysis.}{%
\section{Now that we understand the dataset, we can start the regression
analysis.}\label{now-that-we-understand-the-dataset-we-can-start-the-regression-analysis.}}

\hypertarget{the-first-model-we-will-fit-is-the-5th-degree-polynomial.}{%
\section{The first model we will fit is the 5th degree
polynomial.}\label{the-first-model-we-will-fit-is-the-5th-degree-polynomial.}}

poli5 \textless-
lm(did\textasciitilde hih+I(hih\textsuperscript{2)+I(hih}3)+I(hih\textsuperscript{4)+I(hih}5),tree\_scaling)
summary(poli5)

tree\_scaling \textless- tree\_scaling \%\textgreater\% mutate(di\_poli
= predict(poli5)*dbh)

poli\_rmse \textless- tree\_scaling \%\textgreater\% summarise(RMSE =
sqrt(sum((di\_poli-di)\^{}2)/mean(di\_poli))) \%\textgreater\%
pull(RMSE) \%\textgreater\% round(2)

ggplot(tree\_scaling,aes(x=hih))+ geom\_point(aes(y =
(di\_poli-di)/di\_poli*100))+ geom\_hline(aes(yintercept = 0))+
scale\_y\_continuous(limits=c(-60,60), breaks = seq(-100,100,20))+
scale\_x\_continuous(limits=c(0,1))+ labs(x = `hi / h', y = `Residuals
(\%)', title = `5th degree polynomial taper function (Schöepfer, 1966)',
subtitle = `Dispersion of residuals along the stem', caption =
paste0(`Root Mean Squared Error =', poli\_rmse,`\%'))+
theme(plot.title.position = `plot')

\hypertarget{the-5th-degree-polynomial-is-a-fixed-form-taper-function-that-represents-the}{%
\section{The 5th degree polynomial is a fixed-form taper function that
represents
the}\label{the-5th-degree-polynomial-is-a-fixed-form-taper-function-that-represents-the}}

\hypertarget{average-shape-of-the-stem-profiles-used-to-fit-the-model.}{%
\section{average shape of the stem profiles used to fit the
model.}\label{average-shape-of-the-stem-profiles-used-to-fit-the-model.}}

\hypertarget{for-this-dataset-the-root-mean-square-error-of-this-model-was-3.01-and}{%
\section{For this dataset, the Root Mean Square Error of this model was
3.01\%
and}\label{for-this-dataset-the-root-mean-square-error-of-this-model-was-3.01-and}}

\hypertarget{we-can-see-that-the-residues-are-heteroskedastic.}{%
\section{we can see that the residues are
heteroskedastic.}\label{we-can-see-that-the-residues-are-heteroskedastic.}}

\hypertarget{lets-see-if-we-can-do-better-with-the-bi-model.due-to-its-non-linear-nature}{%
\section{Let's see if we can do better with the Bi model.Due to its
non-linear
nature,}\label{lets-see-if-we-can-do-better-with-the-bi-model.due-to-its-non-linear-nature}}

\hypertarget{we-will-use-the-nlslm-function-from-the-minpack.lm-package-to-estimate-the}{%
\section{we will use the nlsLM function from the minpack.lm package to
estimate
the}\label{we-will-use-the-nlslm-function-from-the-minpack.lm-package-to-estimate-the}}

\hypertarget{model-parameters.}{%
\section{model parameters.}\label{model-parameters.}}

install.packages(``minpack.lm'') library(minpack.lm)

bi \textless- nlsLM(di \textasciitilde{} taper\_bi(dbh, h, hih, b0, b1,
b2, b3, b4, b5, b6), data=tree\_scaling,
start=list(b0=1.8,b1=-0.2,b2=-0.04,b3=-0.9,b4=-0.0006,b5=0.07,b6=-.14))
summary(bi)

tree\_scaling \textless- tree\_scaling \%\textgreater\% mutate(di\_bi =
predict(bi))

bi\_rmse \textless- tree\_scaling \%\textgreater\% summarise(RMSE =
sqrt(sum((di\_bi-di)\^{}2)/mean(di\_bi))) \%\textgreater\% pull(RMSE)
\%\textgreater\% round(2)

ggplot(tree\_scaling,aes(x=hih))+ geom\_point(aes(y =
(di\_bi-di)/di\_bi*100))+ geom\_hline(aes(yintercept = 0))+
scale\_y\_continuous(limits=c(-60,60), breaks = seq(-100,100,20))+
scale\_x\_continuous(limits=c(0,1))+ labs(x = `hi / h', y = `Residuals
(\%)', title = `Bi (2000) trigonometric variable-form taper function',
subtitle = `Dispersion of residuals along the stem', caption =
paste0(`Root Mean Squared Error =', bi\_rmse,`\%'))+
theme(plot.title.position = `plot')

\hypertarget{the-bi-model-performed-better-than-the-polynomial-function-based-on-the-rmse-value.}{%
\section{The Bi model performed better than the polynomial function,
based on the RMSE
value.}\label{the-bi-model-performed-better-than-the-polynomial-function-based-on-the-rmse-value.}}

\hypertarget{however-we-still-have-heteroscedasticity-in-the-residues.-lets-see-what-we-get-by}{%
\section{However, we still have heteroscedasticity in the residues.
Let's see what we get
by}\label{however-we-still-have-heteroscedasticity-in-the-residues.-lets-see-what-we-get-by}}

\hypertarget{the-kozak-2004-model.-we-will-treat-the-p-parameter-of-this-model-as}{%
\section{\texorpdfstring{\adjusting the Kozak (2004) model. We will
treat the p parameter of this model
as}{the Kozak (2004) model. We will treat the p parameter of this model as}}\label{the-kozak-2004-model.-we-will-treat-the-p-parameter-of-this-model-as}}

\hypertarget{one-more-to-be-estimated-using-the-nlslm-function.}{%
\section{one more to be estimated using the nlsLM
function.}\label{one-more-to-be-estimated-using-the-nlslm-function.}}

kozak \textless- nlsLM(di \textasciitilde{} taper\_kozak(dbh, h, hih,
b0, b1, b2, b3, b4, b5, b6, b7, b8, p),
start=list(b0=1.00,b1=.97,b2=.03,b3=.49,b4=-
0.87,b5=0.50,b6=3.88,b7=0.03,b8=-0.19, p = .1), data = tree\_scaling,
control = nls.lm.control(maxiter = 1000, maxfev = 2000) ) summary(kozak)

tree\_scaling \textless- tree\_scaling \%\textgreater\% mutate(di\_kozak
= predict(kozak))

kozak\_rmse \textless- tree\_scaling \%\textgreater\% summarise(RMSE =
sqrt(sum((di\_kozak-di)\^{}2)/mean(di\_kozak))) \%\textgreater\%
pull(RMSE) \%\textgreater\% round(2)

ggplot(tree\_scaling, aes(x=hih))+ geom\_point(aes(y =
(di\_kozak-di)/di\_kozak*100))+ geom\_hline(aes(yintercept = 0))+
scale\_y\_continuous(limits=c(-100,100), breaks = seq(-100,100,20))+
scale\_x\_continuous(limits=c(0,1))+ labs(x = `hi / h', y = `Residuals
(\%)', title = `Kozak (2004) variable-form taper function', subtitle =
`Dispersion of residuals along the stem', caption = paste0(`Root Mean
Squared Error =', kozak\_rmse,`\%'))+ theme(plot.title.position =
`plot')

\hypertarget{by-fitting-the-kozak-2004-model-we-obtained-a-lower-rmse-and-also-managed-to}{%
\section{By fitting the Kozak (2004) model, we obtained a lower RMSE and
also managed
to}\label{by-fitting-the-kozak-2004-model-we-obtained-a-lower-rmse-and-also-managed-to}}

\hypertarget{homogenize-the-dispersion-of-the-residues.}{%
\section{homogenize the dispersion of the
residues.}\label{homogenize-the-dispersion-of-the-residues.}}

\hypertarget{section-2}{%
\section{}\label{section-2}}

\hypertarget{using-taper-models}{%
\section{Using taper models}\label{using-taper-models}}

\hypertarget{in-the-previous-section-we-adjusted-the-three-models-that-have-auxiliary-functions}{%
\section{In the previous section we adjusted the three models that have
auxiliary
functions}\label{in-the-previous-section-we-adjusted-the-three-models-that-have-auxiliary-functions}}

\hypertarget{implemented-in-the-timber-package.-now-lets-explore-the-functions-that-allow-us}{%
\section{implemented in the timbeR package. Now, let's explore the
functions that allow
us}\label{implemented-in-the-timber-package.-now-lets-explore-the-functions-that-allow-us}}

\hypertarget{to-apply-the-fitted-models-in-practice.}{%
\section{to apply the fitted models in
practice.}\label{to-apply-the-fitted-models-in-practice.}}

dbh \textless- 25 h \textless- 20

\hypertarget{all-auxiliary-functions-have-the-argument-coef-where-a-vector}{%
\section{All auxiliary functions have the argument coef, where a
vector}\label{all-auxiliary-functions-have-the-argument-coef-where-a-vector}}

\hypertarget{containing-the-fitted-coefficients-of-the-model-must-be-declared.}{%
\section{containing the fitted coefficients of the model must be
declared.}\label{containing-the-fitted-coefficients-of-the-model-must-be-declared.}}

\hypertarget{this-vector-can-be-accessed-by-using-the-base-r-function-coef.}{%
\section{This vector can be accessed by using the base R function
coef.}\label{this-vector-can-be-accessed-by-using-the-base-r-function-coef.}}

\hypertarget{for-the-kozak-2004-model-we-will-separate-the-p-parameter-from-the}{%
\section{For the Kozak (2004) model, we will separate the p parameter
from
the}\label{for-the-kozak-2004-model-we-will-separate-the-p-parameter-from-the}}

\hypertarget{others.}{%
\section{others.}\label{others.}}

coef\_poli \textless- coef(poli5) coef\_bi \textless- coef(bi)
coef\_kozak \textless- coef(kozak){[}-10{]} p\_kozak \textless-
coef(kozak){[}10{]}

\hypertarget{now-we-can-estimate-the-diameter-di-at-a-given-height-hi.}{%
\section{Now we can estimate the diameter (di) at a given height
(hi).}\label{now-we-can-estimate-the-diameter-di-at-a-given-height-hi.}}

\hypertarget{lets-assume-hi-15-for-this-example.}{%
\section{Let's assume hi = 15 for this
example.}\label{lets-assume-hi-15-for-this-example.}}

hi \textless- 15

poly5\_di(dbh, h, hi, coef\_poli) \#\textgreater{} {[}1{]} 9.224517
bi\_di(dbh, h, hi, coef\_bi) \#\textgreater{} {[}1{]} 8.559173
kozak\_di(dbh, h, hi, coef\_kozak, p = p\_kozak) \#\textgreater{}
{[}1{]} 8.92263 \#\textgreater{} \#Note that there is some variation
between the predictions of the models \# We can better observe this
effect by modeling the complete profile \# of our example tree.

hi \textless- seq(0.1,h,.1)

ggplot(mapping=aes(x=hi))+ geom\_line(aes(y=poly5\_di(dbh, h, hi,
coef\_poli), linetype = `5th degree polynomial'))+
geom\_line(aes(y=bi\_di(dbh, h, hi, coef\_bi), linetype = `Bi (2000)'))+
geom\_line(aes(y=kozak\_di(dbh, h, hi, coef\_kozak, p\_kozak), linetype
= `Kozak (2004)'))+ scale\_linetype\_manual(name = `Fitted models',
values = c(`solid',`dashed',`dotted'))+ labs(x = `hi (m)', y =
`Predicted di (cm)')

\hypertarget{for-the-prediction-of-the-height-at-which-a-given-diameter-occurs-the}{%
\section{For the prediction of the height at which a given diameter
occurs
the}\label{for-the-prediction-of-the-height-at-which-a-given-diameter-occurs-the}}

\hypertarget{procedure-is-similar-to-the-one-presented-above-but-this-time-we-must}{%
\section{procedure is similar to the one presented above, but this time
we
must}\label{procedure-is-similar-to-the-one-presented-above-but-this-time-we-must}}

\hypertarget{declare-the-argument-di-instead-of-hi-for-the-corresponding-functions.}{%
\section{declare the argument di instead of hi, for the corresponding
functions.}\label{declare-the-argument-di-instead-of-hi-for-the-corresponding-functions.}}

di \textless- 10

poly5\_hi(dbh, h, di, coef\_poli) \#\textgreater{} {[}1{]} 14.40821
bi\_hi(dbh, h, di, coef\_bi) \#\textgreater{} {[}1{]} 14.09805
kozak\_hi(dbh, h, di, coef\_kozak, p\_kozak) \#\textgreater{} {[}1{]}
14.2817

\hypertarget{for-this-example-the-application-of-the-three-models-resulted-in-very}{%
\section{For this example the application of the three models resulted
in
very}\label{for-this-example-the-application-of-the-three-models-resulted-in-very}}

\hypertarget{similar-predictions.}{%
\section{similar predictions.}\label{similar-predictions.}}

\hypertarget{the-functions-for-estimating-total-and-partial-volumes-are-similar-to}{%
\section{The functions for estimating total and partial volumes are
similar
to}\label{the-functions-for-estimating-total-and-partial-volumes-are-similar-to}}

\hypertarget{those-presented-so-far-with-some-additional-arguments.-the-following}{%
\section{those presented so far, with some additional arguments. The
following}\label{those-presented-so-far-with-some-additional-arguments.-the-following}}

\hypertarget{procedures-calculate-the-volume-of-the-entire-stem.}{%
\section{procedures calculate the volume of the entire
stem.}\label{procedures-calculate-the-volume-of-the-entire-stem.}}

poly5\_vol(dbh, h, coef\_poli) \#\textgreater{} {[}1{]} 0.414718
bi\_vol(dbh, h, coef\_bi) \#\textgreater{} {[}1{]} 0.4128356
kozak\_vol(dbh, h, coef\_kozak, p\_kozak) \#\textgreater{} {[}1{]}
0.413102 \# We can also estimate partial volumes by declaring the
initial height \# h0 and the final height hi.

hi = 15 h0 = .2

poly5\_vol(dbh, h, coef\_poli, hi, h0) \#\textgreater{} {[}1{]}
0.3884416 bi\_vol(dbh, h, coef\_bi, hi, h0) \#\textgreater{} {[}1{]}
0.3901346 kozak\_vol(dbh, h, coef\_kozak, p\_kozak, hi, h0)
\#\textgreater{} {[}1{]} 0.3863585

\hypertarget{finally-we-will-see-how-the-three-models-estimate-the-volume-and}{%
\section{Finally, we will see how the three models estimate the volume
and}\label{finally-we-will-see-how-the-three-models-estimate-the-volume-and}}

\hypertarget{quantity-of-logs-from-different-wood-products.-we-start-by-defining}{%
\section{quantity of logs from different wood products. We start by
defining}\label{quantity-of-logs-from-different-wood-products.-we-start-by-defining}}

\hypertarget{the-assortments.}{%
\section{the assortments.}\label{the-assortments.}}

\hypertarget{the-assortment-table-must-contain-five-columns-in-order-the-product}{%
\section{The assortment table must contain five columns, in order: the
product}\label{the-assortment-table-must-contain-five-columns-in-order-the-product}}

\hypertarget{name-the-log-diameter-at-the-small-end-cm-the-minimum-length-m}{%
\section{name, the log diameter at the small end (cm), the minimum
length
(m),}\label{name-the-log-diameter-at-the-small-end-cm-the-minimum-length-m}}

\hypertarget{the-maximum-length-m-and-the-loss-resulting-from-cutting-each-log}{%
\section{the maximum length (m), and the loss resulting from cutting
each
log}\label{the-maximum-length-m-and-the-loss-resulting-from-cutting-each-log}}

\hypertarget{cm.-lets-transcribe-the-following-table-into-a-data.frame.-a-point}{%
\section{(cm). Let's transcribe the following table into a data.frame. A
point}\label{cm.-lets-transcribe-the-following-table-into-a-data.frame.-a-point}}

\hypertarget{of-attention-is-that-the-wood-products-must-be-ordered-in-the}{%
\section{of attention is that the wood products must be ordered in
the}\label{of-attention-is-that-the-wood-products-must-be-ordered-in-the}}

\hypertarget{data.frame-from-the-most-valuable-to-the-least-valuable-in-order-to}{%
\section{data.frame from the most valuable to the least valuable, in
order
to}\label{data.frame-from-the-most-valuable-to-the-least-valuable-in-order-to}}

\hypertarget{give-preference-to-the-products-of-highest-commercial-value.}{%
\section{give preference to the products of highest commercial
value.}\label{give-preference-to-the-products-of-highest-commercial-value.}}

assortments \textless- data.frame( NAME = c(`\textgreater{} 15',`4-15'),
SED = c(15,4), MINLENGTH = c(2.65,2), MAXLENGTH = c(2.65,4.2), LOSS =
c(5,5) )

\hypertarget{estimate-volume-and-quantity-of-wood-products-in-a-tree-stem}{%
\section{estimate volume and quantity of wood products in a tree
stem}\label{estimate-volume-and-quantity-of-wood-products-in-a-tree-stem}}

poly5\_logs(dbh, h, coef\_poli, assortments)

bi\_logs(dbh, h, coef\_bi, assortments)

kozak\_logs(dbh, h, coef\_kozak, p\_kozak, assortments)

\hypertarget{there-are-several-additional-arguments-in-the-log-volumequantity-estimation}{%
\section{There are several additional arguments in the log
volume/quantity
estimation}\label{there-are-several-additional-arguments-in-the-log-volumequantity-estimation}}

\hypertarget{functions-that-change-the-way-the-calculations-are-performed.-it-is-highly}{%
\section{functions that change the way the calculations are performed.
It is
highly}\label{functions-that-change-the-way-the-calculations-are-performed.-it-is-highly}}

\hypertarget{recommended-that-you-read-the-functions-help-to-understand-all-its-functionality.}{%
\section{recommended that you read the function's help to understand all
its
functionality.}\label{recommended-that-you-read-the-functions-help-to-understand-all-its-functionality.}}

\hypertarget{an-additional-feature-of-the-timber-package-is-the-possibility-to-visualize}{%
\section{An additional feature of the timbeR package is the possibility
to
visualize}\label{an-additional-feature-of-the-timber-package-is-the-possibility-to-visualize}}

\hypertarget{how-the-processing-of-trees-is-performed-by-the-logs-estimation-functions.}{%
\section{how the processing of trees is performed by the logs estimation
functions.}\label{how-the-processing-of-trees-is-performed-by-the-logs-estimation-functions.}}

\hypertarget{the-arguments-of-these-functions-are-practically-the-same-arguments-of-the}{%
\section{The arguments of these functions are practically the same
arguments of
the}\label{the-arguments-of-these-functions-are-practically-the-same-arguments-of-the}}

\hypertarget{functions-presented-above.}{%
\section{functions presented above.}\label{functions-presented-above.}}

poly5\_logs\_plot(dbh, h, coef\_poli, assortments)

bi\_logs\_plot(dbh, h, coef\_bi, assortments)

kozak\_logs\_plot(dbh, h, coef\_kozak, p\_kozak, assortments)

\hypertarget{using-timber-functions-at-forest-inventory-scale}{%
\section{Using timbeR functions at forest inventory
scale}\label{using-timber-functions-at-forest-inventory-scale}}

\hypertarget{log-estimation-functions-are-performed-one-tree-at-a-time.}{%
\section{Log estimation functions are performed one tree at a
time.}\label{log-estimation-functions-are-performed-one-tree-at-a-time.}}

\hypertarget{applying-these-functions-to-multiple-trees-can-be-performed-in-different-ways.}{%
\section{Applying these functions to multiple trees can be performed in
different
ways.}\label{applying-these-functions-to-multiple-trees-can-be-performed-in-different-ways.}}

\hypertarget{below-are-some-examples-using-the-base-r-function-mapply-and-using-pmap-function}{%
\section{Below are some examples using the base R function mapply and
using pmap
function}\label{below-are-some-examples-using-the-base-r-function-mapply-and-using-pmap-function}}

\hypertarget{from-purrr-package.}{%
\section{from purrr package.}\label{from-purrr-package.}}

\hypertarget{using-mapply}{%
\section{Using mapply}\label{using-mapply}}

tree\_data \textless- data.frame(dbh = c(18.3, 23.7, 27.2, 24.5, 20,
19.7), h = c(18, 24, 28, 24, 18.5, 19.2))

assortment\_vol \textless- mapply( poly5\_logs, dbh =
tree\_data\(dbh,  h = tree_data\)h, SIMPLIFY = T, MoreArgs = list( coef
= coef\_poli, assortments = assortments, stump\_height = 0.2,
total\_volume = T, only\_vol = T ) ) \%\textgreater\% t()

assortment\_vol

\hypertarget{binding-tree_data-and-volumes-output}{%
\section{Binding tree\_data and volumes
output}\label{binding-tree_data-and-volumes-output}}

library(tidyr)

cbind(tree\_data, assortment\_vol) \%\textgreater\% unnest()

library(purrr)

tree\_data \%\textgreater\% mutate(coef = list(coef\_poli), assortments
= list(assortments), stump\_height = 0.2, total\_volume = T, only\_vol =
T) \%\textgreater\% mutate(assortment\_vol = pmap(.,poly5\_logs))
\%\textgreater\% select(dbh, h, assortment\_vol) \%\textgreater\%
unnest(assortment\_vol)

\end{document}
